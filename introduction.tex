%----導入部分---- %
\chapter{Introduction}
\label{sec:intro}
 Images captured under low light conditions suffer from poor visibility, low contrast, and unexpected noise. 
 It is difficult for human to extract hidden meaningful information from these images.
 Moreover, these degradation affect vision techniques, such as consumer digital camera, mobile phones and video surveillance system, which can be often used in outdoors. 
Therefore, many image enhancement methods, including histogram equalization (HE) algorithms \cite{he1} - \cite{he3}, dehaze-based algorithms \cite{haze1}, \cite{haze2}, and Retinex-based algorithms\cite{ssr} - \cite{rrm} have been proposed to improve above problems. \par

%----Histogram Equalization Method---- %
HE algorithms are probably the most intuitive and simplest way to improve image contrast.  The operation stretches the dynamic range of intensity level of an observed image. However, these methods may result in over-enhancement. In addition, these methods cannot consider the intensive noise hidden under dark regions in low light images. Thus, these methods lead to noise amplification after contrast enhancement.\par

%----Dehazing Method---- %
Some researches \cite{haze1}, \cite{haze2} noticed that the inverted low-light images look like haze images. 
According to this observation, these methods have applied to deal with low-light images. However, these methods can obtain reasonable results, they has not provided a physical explanation for the basic model.

%----Retinex Model---- %
Retinex-based methods have been proposed for color images based on human visual system (HVS). 
According to the basic assumption of the Retinex theory \cite{retinex}, an observed image can be decomposed into two parts: the reflectance and illumination components. 
Early attempts in the field, such as Single-Scale Retinex (SSR) \cite{ssr} and Multi-Scale Retinex (MSR) \cite{msr}, treat the reflectance component as the enhancement result. 
However, these methods cause over-enhancement and generate unrealistic result. This problem is mainly caused by the logarithmic operation.
To overcome this problem, Fu proposed proposed simultaneous reflectance and illumination estimation (SRIE) \cite{srie} and a weighted variation model (WVM) \cite{wvm}, and demonstrated that the linear domain model is better than the log-transformed domain in preserving naturalness. 
These methods have good performances in the enhancement, but they still have the problem that noise is quite observable in the results, especially when an observed image has much noise in dark regions. 
By considering the properties of 3D objects, Cai $et$ $al$. \cite{jiep} proposed a Joint intrinsic-extrinsic Prior (JieP) model for Retinex decomposition. However, the model is prone to over-smooth both the reflectance and illumination, since the model cannot sufficiently consider the constraint term about the reflectance. 
Moreover, Li presented a robust Retinex model (RRM) \cite{rrm} by adding a noise term to the cost function to deal with low light image enhancement under hidden noise. RRM suppresses noise amplification effectively, but it has difficulty balancing piecewise smoothed illumination and the structure of the reflectance. 

Although many above methods adopt the $L_{1}$ norm to the reflectance, the tiny details of the estimated reflectance are susceptible to be damaged. Furthermore, by adopting the $L_{2}$ norm to the illumination, the estimated illumination over-smooths and loses the structure information.

In this paper, we propose a new joint optimization equation that considers the features of both reflectance and illumination. We adopt $L_{2}$-$L_{p}$ norm regularization terms to estimate the reflectance as much as possible to preserving tiny details, and to estimate the illumination as much as possible to keep the structure information while avoiding texture-copy problem. Moreover, we introduce a new adaptive texture prior to handle much noise hidden in dark regions and to reveal details and textures of the estimated reflectance. Finally, in the experimental results including qualitative and quantitative evaluations, we show that the proposed method is more effective than the other several state-of-the-art methods.\par