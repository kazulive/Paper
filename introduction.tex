%----導入部分---- %
\chapter{Introduction}
\label{sec:intro}
 Images captured under low light conditions suffer from poor visibility, low contrast, and unexpected noise. Consequently, low-light images prevent human from extracting hidden meaningful information. Moreover, such degradation affect vision techniques, including consumer digital cameras, mobile phones and video surveillance systems, which can be often used in outdoors. Therefore, the methods of low-light image enhancement, including histogram equalization (HE) algorithms \cite{he1} - \cite{he3}, dehaze-based algorithms \cite{haze1}, \cite{haze2}, and Retinex-based algorithms\cite{ssr} - \cite{rrm} have been proposed to deal with above problems. \par
%----Histogram Equalization Method---- %
HE algorithms are probably the most intuitive and simplest way to improve image contrast. The operation stretches the dynamic range of intensity level of an observed image. As a result, the methods tend to result in over-enhancement. In addition, the methods can not consider the intensive noise hidden under dark regions in low light images. Therefore, the methods lead to noise amplification after contrast enhancement.\par
%----Dehazing Method---- %
Some methods \cite{haze1}, \cite{haze2} noticed that the inverted low-light images look like haze images. According to this observation, the methods attempted to deal with low-light images. Although the methods can obtain reasonable results, they do not provide a sufficient physical explanation for the basic model.
%----Retinex Model---- %
Many methods that focus on Retinex theory, which is a color perception model based on human visual system, have been proposed for low-light image enhancement. According to the basic assumption of Retinex theory \cite{retinex}, an observed image can be decomposed into two parts: reflectance and illumination. Early attempts in this theory, such as Single-Scale Retinex (SSR) \cite{ssr} and Multi-Scale Retinex (MSR) \cite{msr}, treat reflectance as the enhancement result. However, the methods cause over-enhancement and generate unrealistic result. This problem is mainly caused by the logarithmic operation when enhancing a low-light image. To overcome this problem, Fu proposed a simultaneous reflectance and illumination estimation (SRIE) \cite{srie} and a weighted variation model (WVM) \cite{wvm}. The methods demonstrated that the linear domain model is better than the log-transformed domain in preserving naturalness. These methods have good performances in the enhancement, but they still have the problem that noise is quite observable in the results, especially when an observed image has much noise in dark regions. Cai $et$ $al$. \cite{jiep} proposed a Joint intrinsic-extrinsic Prior (JieP) model for Retinex decomposition by considering the properties of 3D objects. The method can significantly distinguish between texture and structure regions and estimate illumination while keeping the structure information. However, the method generates noise amplification in dark regions, since the method can not sufficiently consider the constraint term on reflectance. Li presented a robust Retinex model (RRM) \cite{rrm} by adding a noise term to the cost function in order to deal with noise amplification due to the estimation errors. RRM effectively suppresses noise amplification, but the method has difficulty balancing piece-wise smoothed illumination and details of reflectance. Many above methods adopt a $L_{1}$ norm to the constraint term on reflectance, but the tiny details of the estimated reflectance are susceptible to be damaged. Moreover, by adopting a $L_{2}$ norm to the constraint term on illumination, the estimated illumination is over-smoothed without keeping the structure information. \par
In this paper, we propose a new cost function that further considers the characteristics of both reflectance and illumination. We adopt $L_{2}$ and $L_{p}$ norms to the constraint terms on reflectance and illumination in order to preserve fine textures detail as much as possible, and smooth illumination as as much as possible while keeping the structure information. Moreover, we introduce an adaptive texture map into a weight of the constraint term on reflectance. The adaptive texture map is used to tune a noise reduction rate according to brightness of an observed image. In addition, the adaptive texture map contributes to reveal fine textures detail in the estimated reflectance. Finally, we show that the proposed method outperforms the state-of-the-art methods in both qualitative and quantitative evaluations.\par