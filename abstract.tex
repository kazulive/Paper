%----Abstruct---- %
\begin{center}
\section*{Abstract}
\end{center}
The methods of low-light image enhancement have been paid attention to with the wide spread of computer vision applications used in outdoors. Many methods have used cost functions which are based on Retinex theory that decomposes an observed image into reflectance and illumination. However, the enhanced image causes negative effects such as halo artifacts, noise amplification, and over-enhancement when the methods can not sufficiently consider the characteristics of reflectance and illumination. Thus, in order to alleviate such effects, we incorporate constraint terms which further consider the characteristics of reflectance and illumination with a cost function. In addition, we develop an adaptive texture map as a weight of the constraint term on reflectance, which is used to tune noise reduction rate according to brightness of an observed image. Moreover, the adaptive texture map contributes to reveal fine textures detail in the estimated reflectance. Both qualitative and quantitative evaluations show that the proposed method can sufficiently enhance low-light images while suppressing halo artifacts, noise amplification, and over-enhancement.